\documentclass[12pt]{article}

\usepackage[a4paper, inner=2.5cm, outer=2.5cm, top=2.5cm, bottom=2cm, bindingoffset=0cm]{geometry}

\title{\Large{\textbf{3D-Tracking-System auf Basis herkömmlicher Kameras}}}
\author{Alexander Minor}

% text input
\usepackage[utf8]{inputenc}
\newcommand\Kurzfassung{
    Das Ziel unseres Projektes ist es, eine robuste, zuverlässige, modulare, 3D-gedruckte und einfach zu bedienende Unterwassersonde zu designen und mit einem elektrischen System auszustatten, welches mit diversen Sensoren Daten für die spätere graphische Visualisierung und Analyse sammelt. Wir erarbeiteten zunächst verschiedene Konzepte, kombinierten ihre Vorteile und begannen mit dem Entwicklungsprozess der Sonde. Im Laufe des Projektes bauten wir mit Hilfe von zahlreichen Tests und Prototypen ein fertiges Uboot, die wir erfolgreich in einem Gewässer testen konnten. 
}
\newcommand\Einleitung{
    Bemannte Tauchgänge in die Tiefe von Gewässern sind meist umständlich, kompliziert, bergen ein hohes Risiko und sind vor allem teuer. Trotzdem ist es in vielen Situationen in Wirtschaft und Forschung notwendig, dass gewisse Aktionen und Datenmessungen tief unter dem Meeresspiegel stattfinden. Deshalb kommen immer mehr moderne Tauchsonden oder auch Tauchroboter zum Einsatz, welche je nach Notwendigkeit den Druck aushalten und somit auch wasserfest sein müssen, um Elektronik und Messgeräte in dieser Umgebung zu schützen. Diese Tauchsonden können meist autonom oder ferngesteuert diese Aufgaben bewerkstelligen, sodass die Entwicklung dieser in den letzten Jahren und Jahrzehnten immer mehr an Relevanz gewonnen hat.
}

\newcommand\StandEntwicklung{
Unterwassersonden und Drohnen haben und werden auch in der Zukunft noch größeren Nutzen in Wirtschaft und Forschung haben. \\
Die Meeresböden der Ozeane in bestimmten Regionen bergen das Potenzial einer wichtigen Rohstoffquelle für wichtige Erze wie Kobalt, Kupfer und Nickel, welche hohe Nachfrage erfahren, da sie essenziell für moderne Technik sind und derzeitige Quellen aufgrund schlechter Arbeitsbedingungen und Ausbeutung an Attraktivität verlieren. In diesen Gebieten lassen sich tief unter der Wasseroberfläche ganze Felder an sogenannten Manganknollen finden, ballähnliche Metallklumpen, welche unter anderem die genannten Rohstoffe enthalten. Bisher war der „Abbau“ jedoch aus Umweltschutzgründen nicht möglich, da vorherige Abbaumethoden das Ökosystem jahrzehntelang nachhaltig schädigen würden. Nun gibt es aber neue Versuche, vor allem durch die wachsende Nachfrage motiviert, Möglichkeiten zu finden, eine umweltschützende Ernte zu ermöglichen. Dafür werden wiederum ferngesteuerte und autonome Unterwasser-Drohnen und Sonden verwendet, welche im Netzwerk mit Abbaumaschinerie den Boden abscannen und überwachen, sodass der Abbau stattfinden kann, ohne dass beheimatete Tier- und Pflanzenarten bedroht oder die Sedimentschichten als Lebensraum zerstört werden. Gerade wird getestet, wie stark diese Methode den Lebensraum beeinflusst. \\
Weiterhin können Unterwassersonden und Roboter bei der Untersuchung und Reparatur von Offshore-Windanlagen eine große Erleichterung bieten. Besonders vor dem Hintergrund des Ausbaus der erneuerbaren Energien und der folglich steigenden Anzahl an Windparks vor den Küsten wird die Instandhaltung dieser Anlagen immer wichtiger. Dabei sind Effizienz und Vereinfachung wichtige Faktoren. So können die Anlagen beispielsweise durch Kameras und Positionssensorik 3D-visualisiert werden und auf Schäden untersucht werden. Dadurch können aufwendige Tauchgänge durch Menschen ersetzt werden. Selbst Reparaturen können durch Roboter ferngesteuert durchgeführt werden. \\
Dies sind nur zwei Anwendungsbeispiele, an denen man sieht, dass Unterwassertechnik ein immer wichtigeres Thema wird und daher die Forschung- und Entwicklungsarbeit in diesem Bereich bedeutsamen Fortschritt bringt.
}

\newcommand\Ziele{
    Wir planen, nicht nur einfach eine Sonde zu bauen, sondern eine Plattform für Schüler und Schülerinnen unserer Schule, auf der sie in Zukunft auch eigene Experimente durchführen können. Deshalb soll die Bedienung der Sonde auch unkompliziert sein. Eine robuste Konstruktion ist unerlässlich, damit die Sonde öfter verwendet werden kann und während des Tauchgangs stabil bleibt. Hierbei sollen bewegliche Teile zur Vereinfachung und Stabilität so weit wie möglich minimiert werden. \\
    Die Sonde soll die Fähigkeit besitzen, verschiedene Experimente je nach den aktuellen Anforderungen zu transportieren. Durch einen modularen Aufbau wird ermöglicht, dass zusätzliche Module, die eigene Experimente oder Sensorik beinhalten, gebaut und sehr einfach eingebaut werden können. \\
    Es ist wichtig, dass die Sonde die Fähigkeit hat, Elektronik in einem wasserdichten Raum zu transportieren. Die Sonde soll auch groß genug sein, damit die Größe unserer (Elektronik-)Module und auch die der im Nachhinein konstruierten Experimentiermodule wenig eingeschränkt werden. \\
    Die Sonde muss in der Lage sein, in einer Tauchtiefe von ca. 70 Metern unter dem Meeresspiegel in sowohl Süß- als auch Salzwasser zu arbeiten. Sie benötigt nur eine Art Tiefenkontrolle, da keine Bewegung in die anderen Richtungen stattfindet. Während des Tauchvorgangs sollen durch die Elektronik an Bord Daten (Druck, Temperatur, Licht, Neigung, GPS) und Video-/Soundaufnahmen aufgezeichnet, sowie Wasserproben gesammelt werden, die dann zurück an der Oberfläche entnommen und später analysiert werden können, um weitere Informationen über das Gewässer zu erlangen. \\
    Natürlich sind Kosten aber auch Nachhaltigkeit ein entscheidender Faktor bei der Konstruktion und der Aufwand für einen einzelnen Tauchgang soll möglichst geringgehalten werden.
}
\newcommand\Thema{
    Ab der zehnten Klasse gab es an unserer Schule das Angebot der freiwilligen AG „JIA“ (Junior Ingenieurs Akademie). Im Rahmen dieser AG wurden verschiedene technische Projekte verwirklicht (z.B. Formel 1 in der Schule). Die Gruppe aus der höheren Klasse, die damals vor uns ein Projekt entwickelte, war an unserer Schule relativ erfolgreich mit ihrem Projekt. Sie hatten damals einen Wetterballon steigen lassen, welcher verschiedene Daten während des Fluges aufnahm (genannt: „KGSgoesStrato“). Nun wurden wir von unseren Lehrern angeregt auch ein derartiges Projekt zu gestalten. Und so mussten wir uns ein passendes Thema für unser Projekt überlegen. Die Überlegung ging dann in die Richtung: „Das vorherige Projekt ging nach oben, wie wäre es, wenn wir jetzt nach unten gehen, vielleicht unter Wasser?“. Die Idee fand schnell Anklang und es wurde auch angemerkt, dass für den Biologieunterricht (und auch Geografieunterricht) oft Gewässerdaten und auch Wasserproben gesucht werden, um diese auszuwerten. So entstand die Idee, eine Sonde zu entwickeln, welche diese Aufgaben und mehr erfüllen kann.
}
\newcommand\KonteptEins{
    Das erste Konzept hatte einen Tauchmechanismus ähnlich dem, der in echten U-Booten verwendet wird, und würde modulare Experimentmodule tragen, die übereinandergestapelt werden.
    Der Tauchgang würde von einem speziellen Modul durchgeführt werden, welches die gesamte für den Tauchgang erforderliche Hardware enthält. Zur Tiefenkontrolle werden ein Wassertank und ein Luft Tank verwendet, die zusammen mit Hilfe computergesteuerter Ventile den Auf- und Abtrieb der Sonde kontrolliert. 
    Verschiedene Experimentiermodule werden übereinandergestapelt und miteinander verbunden. Diese Module können angepasst werden und ermöglichen es, je nach Bedarf wissenschaftliche Experimente zur Sonde hinzuzufügen oder zu entfernen. 
}
% pictues and shit
\usepackage{graphicx}
% define assets path
\graphicspath{{assets/}}
% font carlito (calibri on crack)
\usepackage[sfdefault]{roboto}
% Used to create filler text
\usepackage{blindtext}
% Used to include pictures
\usepackage{graphicx}
% Used to wrap text around pictures
\usepackage{wrapfig}
% Used to compact list es
\usepackage{enumitem}
% Used to customize the page layout of your LaTeX documents
\usepackage{fancyhdr}
% Blocksatz
\usepackage{ragged2e}
% Used to create an index
\usepackage{index}
% Seperate Sections into multiple files
\usepackage{subfiles}
% really dont understand this shit here:
\usepackage[font=small]{caption}
\usepackage{subcaption}
\usepackage{microtype}
\usepackage{wallpaper}
\usepackage{setspace}
\usepackage[export]{adjustbox}
\usepackage[german]{babel}
\usepackage{tocloft}
\usepackage{caption}


\makeindex

\onehalfspacing
% \setcounter{secnumdepth}{5}
% \setlength{\abovecaptionskip}{3pt plus 1pt minus 1pt} 
\setlength{\intextsep}{0pt}
% \setlength{\columnsep}{15pt}
\setlength{\parindent}{0cm}
\setlength{\parskip}{0.3cm}
% \setlength{\belowcaptionskip}{-20pt}
% \captionsetup{belowskip=10pt}
% Background

%%%%%%%%%% Startseite
\begin{document}
\begin{titlepage}
  \begin{center}
    \vspace*{3cm}

    {\LARGE{3D-Tracking System auf Basis herkömmlicher Kameras}}

    \vspace{0.5cm}
    {\normalsize{Alexander Minor}}

    \vspace{3.5cm}

    \raggedright{\textbf{Schule:} Klaus-Groth-Schule Neumünster} \\
    \vspace{0.2cm}
    \raggedright{\textbf{Referenzfach:} Informatik/Mathematik} \\
    \vspace{0.2cm}

  \end{center}
\end{titlepage}

%%%%%%%%%% Inhaltsverzeichnis
\newpage
\tableofcontents

\newpage

\section{Einleitung}
Das genaue Bestimmen der Position eines Objektes im dreidimensionalen Raum ist zu einer grundlegenden technischen Vorrausetzung in vielen Gebieten geworden. Egal ob in Motion-Capture, in der Produktion von Filmen oder im Bereich der Robotik, solche Systeme sind ein essenzielles Werkzeug in der Entwicklung und Evaluierung von Software und physischen Produkten.\par
\begin{wrapfigure}{r}{0.40\textwidth}
    \includegraphics[width=\linewidth]{motion_capture_fifa.jpg}
    \caption{Motion Capture in der Filmproduktion}
    \includegraphics[width=\linewidth]{drones.jpg}
    \caption{Ein Schwarm von Dronen kontroliert durch 3D-Tracking}
\end{wrapfigure}
In Filmen werden 3D-Tracking Systeme verwendet, um Bewegungen, Kämpfe und Szenen digital aufzuzeichnen und die gewonnenen Daten mit Hilfe computergenerierter Effekte zu den spektakulären Szenen heutiger Filme zu verwandeln, ohne die einzigartige und komplexe Bewegungsart des Menschen zu vernachlässigen. Im Sport können Trajektorien eines Balles oder Bewegungen eines Spielers aufgezeichnet werden, um dem Training wichtige Informationen hinzuzufügen und auch in der Robotik ermöglicht das Tracking genau Steuerung und Korrektur von Drohnen oder Robotern. Nur durch die exakte Erfassung der Position können Fehler direkt oder im Voraus erkannt und korrigiert werden, denn auch wenn die Bewegung eines Roboter-Armes für das menschliche Auge korrekt wirken, könnten milimeter genaue Fehler durch ein 3D-Tracking-System erkannt werden. 
\subsection{Problematik}
\begin{wrapfigure}{r}{0.38\textwidth}
\centering
  {\setlength{\belowcaptionskip}{-10pt}
    \includegraphics[angle=0,width=\linewidth]{4143.png}
    \caption{Optitrack UV-Tracking-Kamera}
  }
\end{wrapfigure}
Die Problematik mit dem sich dieses Projekt befasst, ist dass es nur wenige Anbieter für solche Systeme gibt. Die Technik ist sehr teuer und nur gut finanzierte Projekte haben Zugriff auf diese revolutionären Systeme. Der bekannteste Anbieter Optitrack verlangt für die rechts gezeigt Kamera 4143\$, von denen für ein funktionierendes System mindestens 4 gebraucht werden. 

Das Ziel dieses Projektes ist es daher, ein Open-Source, günstiges und einfach zu integrierendes dreidimensionales Trackings-System zu entwickeln, damit Forscher, Entwickler und Interessierte auf der ganzen Welt, egal ob für Robotik, Filmproduktion oder der Messung von Bewegungen, 3D-Tracking als Basis für ihre Projekte verwenden können. \par
In dieser Projektarbeit wird die Theorie,der Entwicklungsprozess und die Ergebnisses meines 3D-Tracking Systems auf Basis herkömmlicher Webcams beschrieben und evaluiert.  

\newpage
\section{Vorgehen}
Bevor die tatsächliche Entwicklung der Software beginnen konnte, mussten zuerst die theoretische Funktionsweise definiert werden und wie der Aufbau des Systemes grob aussehen sollte.

\subsection{Theorie}
In einer Aufnahme einer herkömmlichen Kamera lässt sich die Position eines Objektes im 2d Raum mit Hilfe von Object Tracking erfassen. Erweitert man diese Informationen mit den intrinsischen Werten der Kamera, also Brennweite, Verzerrung und Skalierung, in den 3d Raum, fehlen einem die nötigen Informationen um die Tiefe, also die Entfernung zur Kamera zu berechnen.  

\begin{wrapfigure}{r}{0.38\textwidth}
\centering
  \includegraphics[angle=0,width=\linewidth]{triangulation.png}
  \caption{unterschiedliche Kamera Perspektiven}
\end{wrapfigure}
Es bleibt also eine Gerade, auf der sich das Objekt befinden könnte. Wenn man nun eine zweite Kamera hinzufügt, erhält man eine weitere Gerade, auf der sich das Objekt befinden könnte. Liegen die Kameras an unterschiedlichen Orten, lässt sich die akkurate Position des zu trackenden Objektes durch den Schnittpunkt der beiden Geraden berechnen. Wenn man jegliche Messfehler vernachlässigt, und die exakte Position der Kameras bekannt ist, würden zwei Kameras schon ausreichen, um die nötigen Daten zur Berechnung zu liefern.  

In der Realität gibt es aber einige Aspekte, die die Genauigkeit und Funktionalität des Systems negativ beeinflussen: 
\begin{itemize}
  \item Da ein Bild einer Kamera eine begrenzte Auflösung hat, kann eine Kante eines Objektes nur die Hälfte eines Pixels ausfüllen. Durch das „Auf- oder Abrunden“ dieser Bildinformationen entsteht ein Messfehler, der mit zunehmender Distanz größer wird. 
  \item Da die verwendeten Kameras meist keine Informationen zu den intrinsischen Daten mitliefern, lassen sich diese nur mit Hilfe einer Kalibrierungsequenz ermessen. In der werden mehrere aufgenommen verzerrte Bilder so angepasst, dass sie den tatsächlichen Eigenschaften eines bekannten Objektes nahekommen, in der Praxis ist dies meist ein Schachbrett. In der resultierenden Projektion der Aufnahme im dreidimensionalen Raum können durch Messfehler Verzerrungen oder Verschiebungen entstehen. 
  \item Der mit Abstand stärkste Einfluss auf mögliche Fehler ist die durch eine Kalibrierungssequenz geschätzte Position der Kameras. Sowohl Rotation und Transposition zur Uhrkamera müssen perfekt sein, damit tatsächlich ein Schnittpunkt entsteht. 
\end{itemize}

Es wird also ersichtlich, dass ein perfekter Systemaufbau aufgrund der vielen Variablen nicht realistisch ist, und die Position des getrackten Objektes angenähert werden muss. Um diese möglichst nahe an die tatsächlichen Werte zu bringen, gibt es folgende Möglichkeiten: 
\begin{itemize}
  \item Eine größere und besser verteilte Anzahl an Kameras. Der kleinste Fehler für die Annäherung eines möglichen Schnittpunktes verschiedener Geraden entsteht dann, wenn diese orthogonal zueinander verlaufen. Es ist also vorteilhaft die Kameras möglichst effektiv im Raum zu positionieren. Auch die Anzahl verringert den Fehler, da einerseits mehr Daten entstehen und anderseits beim Ausfall oder starkem Messfehler nicht gleich das ganze System zusammenbricht. 
  \item Ein Algorithmus zum Erkennen von starken Messfehlern und guter Annäherung eines möglichen Schnittpunktes. 
\end{itemize}

\newpage
\subsection{Technische Voraussetzungen}
Bevor es an die tatsächliche Entwicklung gehen kann, müssen einige Spezifikationen, Voraussetzungen und zu verwendende Systeme definiert werden, damit das grobe Konzept auch tatsächlich umsetzbar ist. Dazu gehört die verwendete Hardware in Form von Kameras und Software in der Form von Programmiersprache, Bibliotheken und Programmen zur Entwicklung und Visualisierung. 

\subsubsection{Python}
Da das Projekt komplex ist, ist eine „high-level“ Programmiersprache zu präferieren, wodurch komplexere Aufgaben mit weniger Befehlen gelöst werden können. In diesem Bereich bietet sich Python besonders an, da diese einerseits großen Support in Form von zahlreichen Librarys und einer aktiven Community besitzt und ich andererseits bereits viel Erfahrung in diesem Bereich hatte. 

\subsubsection{ROS}
Das zu entwickelnde System sollte besonders variabel, erweiterbar und vielseitig nutzbar sein, es macht also Sinn den Code modular aufzubauen, so dass bestimmte Teile ausgetauscht, verbessert oder erweitert werden können, ohne die Funktionalität zu zerstören.  

Um eine solche Modularität schaffen zu können, braucht man bestimmte Vereinbarungen und Vorgaben, wie zwischen Teilen kommuniziert werden soll, wie diese voneinander abhängig sind und wie die generelle Struktur des Systems aussieht.  

So wäre es beispielsweise sinnvoll, das Auslesen, Verarbeiten und Vergleich der Kamerabilder voneinander zu separieren, so dass einerseits eine gewisse Unabhängigkeit entsteht und andererseits so von Multithreading nutzen gemacht werden kann, der Aufteilung der Rechenlast auf mehrere CPU-Kerne.  

Auch wenn zahlreiche Lösungen für das Aufteilen eines Python Programmes in verschiedene Threads (Unterprogramme) bereits nativ möglich ist, wäre die Integration in komplexere Systeme äußerst mühselig, da vor allem die Kommunikation in Form von geteilten Variablen eher primitiv und zeitlich aufwändig wäre. Es musste also ein System her, dass die Kommunikation und Separierung der Programme vereinfachen würde und idealerweise bereits vorgefertigte System für kleinere Aspekte des Projektes hätte, so dass bei der Entwicklung auf bereits getesteten Code zurückgegriffen werden konnte. 

\begin{wrapfigure}{r}{0.38\textwidth}
\centering
  \includegraphics[angle=0,width=\linewidth]{ROS_diagram.jpg}
  \caption{Anwendungen von ROS}
\end{wrapfigure}
Alle diese Voraussetzungen erfüllte das Open-Source System „ROS“ (Robotics Operation System), ein seit 2007 entwickelte Framework für Roboter, welches in zahlreichen persönlichen und Industriellen Systemen verwendet wird. ROS, oder im Falle dieses Projektes die zweite Version ROS2, ist weniger eine Library für eine bestimmte Programmiersprache, sondern wie im Namen schon enthalten, ein „Operating System“, also eine Vielzahl an Librarys, Tools und Algorithmen für die Entwicklung von Robotern, die tiefer im System integriert sind. Der Code selbst wird dann in Python geschrieben, und kann mit Hilfe einer Library (rclpy) mit ROS kommunizieren. 

ROS ermöglicht es nun, so genannte „Nodes“ zu erstellen, diese sind unabhängige Module, die in ihrer eigenen Python oder C++ Datei erstellt werden können. So gibt es beispielsweise für jede Kamera eine Node, die sich nur um das Auslesen der Bilder kümmert, eine weitere zum Erkennen verschiedener Objekte und eine andere zum Vergleichen dieser Daten mit anderen Kameras.  

\newpage

\begin{wrapfigure}{r}{0.46\textwidth}
\centering
  \includegraphics[angle=0,width=\linewidth]{Publisher.png}
  \caption{Publisher-Subscriber-Modell}
  \includegraphics[angle=0,width=\linewidth]{Server.png}
  \caption{Server-Client-Modell}
\end{wrapfigure}
Allein können diese Module allerdings nichts erreichen, da sie von Informationen untereinander abhängig sind. Auch hier bietet ROS zwei zentrale Tools zur Kommunikation zwischen Nodes:  

Einmal das sogenannte Publisher-Subscriber Modell, in dem die Kommunikation ähnlich dem UDP-Netzwerkprotokoll verläuft und die Nodes nichts von gegenseitiger Existenz wissen, und einmal des Service Modell, in dem die Kommunikation ähnlich dem TCP-Netzwerkprotokoll verläuft, es also Anfragen (Requests) und Antworten (Responses) gibt.  

Ein weiterer Kernaspekt, weshalb ROS ein essenzieller Part dieser Entwicklung ist, hat mit der Etablierung der Technik zu tun. Da jede Art von Kommunikation durch sogenannte „Messages“ beschrieben wird, also welche Daten genau übertragen werden, gibt es viele Programme, die diese „verstehen“ können. Bilder, Vektoren, Kamera-Matrizen oder simple Punkte, werden daher immer auf dieselbe Weise versendet. So gibt es gute Möglichkeiten zur Visualisierung und Simulation der Daten, sowie zahlreiche andere Apps, die die Entwicklung durch ROS erleichtern. 
\subsubsection{Opencv}
Eine weitere Library die in diesem Projekt eine große Rolle spielt, ist OpenCV, bzw. die Python Implementation. Dies ist eine freie Programmbibliothek mit Algorithmen für die Bildverarbeitung und Computer Vision. Durch sie wird sowohl das Auslesen der Kameradaten, als auch die Kalibrierung zur Berechnung der intrinsischen Parameter möglich.
\subsection{Vorgehensweise und Wandel}
Um die Versitalität und Zuverlässigkeit des Systemes sicherzustellen, wurde in der Entwicklung jedes Modul ausgiebig getestet, sowohl seperat als auch im System integriert. Das Vorgehen in der Programmierung war also stets mit dem Testen des Systemes verbunden, weshalb jedes Modul kontinuierlich im Laufe des Projektes verbessert wurde, sobald Fehler entdeckt wurden oder neu Funktionalität nötig war. 

\newpage
\section{Systemaufbau}
Jede Ausführung des Tracking-Systemes startet mit dem sogenannten "Manager", dies ist ein Programm, welches nach angeschlossenen Kameras sucht und alle nötigen Nodes zum Tracking configuriert und startet. Ein laufendes System kann in etwa so aussehen:
\begin{figure}[h]
  \centering
  \includegraphics[angle=0,width=\linewidth]{anytrack-basic-tracking.png}
  \caption{Nodes, Topics und Messages eines laufenen Systemes mit 2 Kameras}
\end{figure}

\subsection{2D-Tracking}
Bevor die Position eines Objektes im dreidimensionalen Raum berechnet werden kann, müssen erstmal die zweidimensionalen Koordinaten des Objektes im Bild der Kamera errechnet werden. Als Tracking-Objekt dienen kleine hellgrüne Kugeln, da sie von allen Seiten gleich aussehen und gut vom Hintergrund zu unterscheiden sind  

Üblicherweise wird für solche Zwecke Object-Tracking verwendet, bei dem die bestimmten visuellen Eigenschaften eines Objektes gespeichert und im nächsten Frame danach gesucht wird. Ein solches Objekt kann ein Auto, ein Mensch oder in diesem Fall eine Kugel sein.  

Auch wenn Object-Tracking mit Librarys wie OpenCV erstaunlich leicht zu integrieren ist, gibt es in diesem Fall eine simplere und schnellere Methode: Das Tracking von bestimmten Farb-Konturen, die nach dem Anwenden einer bestimmten „Maske“ übrigbleiben. 

Die Problematik bei dieser Technik liegt in der Unterscheidung des Objektes zum Hintergrund. In professionellen, aber sehr viel teuren Systemen, wird auf Infrarot-Kameras und besonders reflektierende kleine Kugeln gesetzt. Dieselbe Technik lässt sich aber auch auf herkömmlichen Webcams anwenden, wenn die Kugeln in einer besonders starken und im Hintergrund nicht vorhandenen Farbe sind. 

\begin{wrapfigure}{r}{0.40\textwidth}
\centering
  \includegraphics[angle=0,width=\linewidth]{rgb-hsv.png}
  \caption{RGB und HSV Farbformat}
  {\setlength{\belowcaptionskip}{0pt}
    \includegraphics[angle=0,width=\linewidth]{2d-normal-filter.jpg}
    \caption{angewandter Farbfilter}
  }
\end{wrapfigure}
Durch das Konvertieren des aufgenommenen Bildes vom RGB zum HSV-Format (siehe Abbildung), kann man die Daten so filtern, dass nur eine bestimmte Farbe übrigbleibt. Das HSV-Format ist dabei sehr nützlich, da ähnliche Farben (wie beispielsweise hell und dunkelgrün), nah einander liegen, und nur durch Filtern des Farbwertes (Hue), der Farbsättigung (Saturation) und dem Hellwert (Value), eine grobe Maske erstellt werden kann. 

\begin{wrapfigure}{r}{0.40\textwidth}
\centering
  \includegraphics[angle=0,width=\linewidth]{2d-normal.jpg}
  \includegraphics[angle=0,width=\linewidth]{2d-normal-dark.jpg}
  \caption{Aufnahme in unterschiedlichen Lichtverhältnisses}
  \includegraphics[angle=0,width=\linewidth]{2d-uv-normal.jpg}
  \caption{Aufnahme mit UV-Licht}
  \includegraphics[angle=0,width=\linewidth]{2d-color.jpg}
  \includegraphics[angle=0,width=\linewidth]{2d-color-dark.jpg}
  \caption{Aufnahme mit Nachbearbeitung in unterschiedlichen Lichtverhältnisses}
\end{wrapfigure}
Beim Testen dieses Konzeptes stellte sich jedoch heraus, dass die Farbe allein nicht genug Kontrast ist, um die Kugel zuverlässig vom Hintergrund zu unterscheiden. Bei schlechter oder einseitiger Beleuchtung wurde eine Seite der Kugel signifikant dunkler, so dass die Filter, die am Tag funktioniert haben, nun nur noch die Hälfte durchlassen würden.  

Die ursprüngliche Lösung für dieses Problem sollte eine UV-Lampe gewesen, da das grüne Plastik der Kugel das umgangssprachlich genannte „Schwarzlicht“ stark reflektiert, und so stark von anderen Objekten im Raum unterschieden werden konnte. 

Auch wenn diese Lösung in gewissen Fällen funktioniert hat, musste die Stärke der Lampe und die Farbfilter für jede Umgebung neu eingestellt werden, um korrekt zu funktionieren. Ebenfalls war die Qualität des Trackings stark von dem Einfallswinkel des UV-Lichtes abhängig war. Würde dieses nämlich direkt in die Linse der Kamera reflektiert werden, was beim Auftreffen auf eine Kugel immer der Fall ist, würde dieser Bereich möglicherweise komplett weiß werden (siehe Abbildung). Anders als Hellgrün ist Weiß eine Farbe, die immer im Hintergrund zu finden ist, und kann daher nicht in die Farbfilter mit integriert werden. 

Eine viel einfachere Lösung ergab sich durch das Experimentieren mit verschiedenen Bildbearbeitungen, die vor dem Tracking Algorithmus auf das Bild angewendet werden.  
Durch das Vermindern des Kontrastes und der Erhöhung der Sättigung sticht die Kugel nun viel stärker aus dem Bild hervor, und kann so deutlich einfacher mit einer Maske ausgeschnitten werden. Zudem verschwindet so der Kontrast zwischen den Seiten der Kugel, wodurch man einen stärker fokussierten Farbfilter verwenden kann, was Messfehler drastisch reduziert.  
\newpage
\begin{wrapfigure}{r}{0.40\textwidth}
  {\setlength{\belowcaptionskip}{-30pt}
    \includegraphics[angle=0,width=\linewidth]{2d-color-filter.jpg}
    \caption{Farb-Maske nach Bildbearbeitungen}
  }
\end{wrapfigure}
Von der nun übrig gebliebenen Binär-Maske lässt sich relativ einfach das Zentrum einer Kontur und der Radius berechnen, indem eine Kreisform an die Konturen der einzelnen Masken angelegt wird. Durch diese Technik kann das Zentrum einer Kugel selbst dann noch akkurat berechnet werden, wenn die Hälfte abgedeckt wird. 

\subsubsection{Identifizierung der Objekt}
Auch wenn es für die meisten Anwendungen nicht nötig ist, können verschiedene Tracking Punkte durch zwei unterschiedliche Methoden im zeitlichen Verlauf und selbst in Bewegung voneinander unterschieden werden. Einerseits können mehrere Tracking-Farben hinzugefügt werden, zum Beispiel eine rote, blaue und grüne Kugel, andererseits vergibt der Tracking-Algorithmus jeder Kontur eine ID, und versucht im nächsten Frame diese anhand der Nähe zu den neuen Ergebnissen zum selben zu zuteilen. In der Praxis funktioniert die zweite Technik trotz ihrer Simplizität erstaunlich gut, sogar beim Überkreuzen zweier Laufbahnen.  

\subsection{Kamera-Kalibrierung}
\subsection{3D-Vektoren}
\subsection{Positions-Kalibrierung}


\section{Ergebnisse}

\end{document}