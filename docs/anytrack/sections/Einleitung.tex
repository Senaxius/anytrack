\documentclass[../main.tex]{subfiles}
\graphicspath{{\subfix{../assets/}}}
\begin{document}
Das genaue Bestimmen der Position eines Objektes im dreidimensionalen Raum ist zu einer grundlegenden technischen Vorrausetzung in den letzten Jahren geworden. Egal ob in Motion Capture Systemen, in der Produktion von Filmen oder im Bereich der Robotik, solche Systeme sind ein essenzielles Werkzeug in der Entwicklung und Evaluierung von Software und physischen Produkte. In Filmen werden 3D-Tracking Systeme verwendet, um Bewegungen, Kämpfe und Szenen digital aufzuzeichnen und die gewonnenen Daten mit Hilfe computergenerierter Effekte zu den unrealen Szenen heutiger Filme zu verwandeln, ohne einzigartige Bewegungsart des Menschen zu vernachlässigen. Im Sport können Trajektorien eines Balles oder Bewegungen eines Spielers aufgezeichnet werden, um dem Training wichtige Informationen hinzuzufügen und auch in der Robotik ermöglicht das Tracking genau Steuerung und Korrektur von Drohnen oder Robotern. Nur durch die exakte Erfassung der Position können Fehler direkt oder im Voraus erkannt und korrigiert werden. Die Bewegung eines Robot-Armes könnten für das menschliche Auge korrekt wirken, durch ein 3D-Tracking-System lässt sich dies auch Millimeter genau testen. 
\end{document}