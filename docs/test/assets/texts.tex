\newcommand\Kurzfassung{
    Das Ziel unseres Projektes ist es, eine robuste, zuverlässige, modulare, 3D-gedruckte und einfach zu bedienende Unterwassersonde zu designen und mit einem elektrischen System auszustatten, welches mit diversen Sensoren Daten für die spätere graphische Visualisierung und Analyse sammelt. Wir erarbeiteten zunächst verschiedene Konzepte, kombinierten ihre Vorteile und begannen mit dem Entwicklungsprozess der Sonde. Im Laufe des Projektes bauten wir mit Hilfe von zahlreichen Tests und Prototypen ein fertiges Uboot, die wir erfolgreich in einem Gewässer testen konnten. 
}
\newcommand\Einleitung{
    Bemannte Tauchgänge in die Tiefe von Gewässern sind meist umständlich, kompliziert, bergen ein hohes Risiko und sind vor allem teuer. Trotzdem ist es in vielen Situationen in Wirtschaft und Forschung notwendig, dass gewisse Aktionen und Datenmessungen tief unter dem Meeresspiegel stattfinden. Deshalb kommen immer mehr moderne Tauchsonden oder auch Tauchroboter zum Einsatz, welche je nach Notwendigkeit den Druck aushalten und somit auch wasserfest sein müssen, um Elektronik und Messgeräte in dieser Umgebung zu schützen. Diese Tauchsonden können meist autonom oder ferngesteuert diese Aufgaben bewerkstelligen, sodass die Entwicklung dieser in den letzten Jahren und Jahrzehnten immer mehr an Relevanz gewonnen hat.
}

\newcommand\StandEntwicklung{
Unterwassersonden und Drohnen haben und werden auch in der Zukunft noch größeren Nutzen in Wirtschaft und Forschung haben. \\
Die Meeresböden der Ozeane in bestimmten Regionen bergen das Potenzial einer wichtigen Rohstoffquelle für wichtige Erze wie Kobalt, Kupfer und Nickel, welche hohe Nachfrage erfahren, da sie essenziell für moderne Technik sind und derzeitige Quellen aufgrund schlechter Arbeitsbedingungen und Ausbeutung an Attraktivität verlieren. In diesen Gebieten lassen sich tief unter der Wasseroberfläche ganze Felder an sogenannten Manganknollen finden, ballähnliche Metallklumpen, welche unter anderem die genannten Rohstoffe enthalten. Bisher war der „Abbau“ jedoch aus Umweltschutzgründen nicht möglich, da vorherige Abbaumethoden das Ökosystem jahrzehntelang nachhaltig schädigen würden. Nun gibt es aber neue Versuche, vor allem durch die wachsende Nachfrage motiviert, Möglichkeiten zu finden, eine umweltschützende Ernte zu ermöglichen. Dafür werden wiederum ferngesteuerte und autonome Unterwasser-Drohnen und Sonden verwendet, welche im Netzwerk mit Abbaumaschinerie den Boden abscannen und überwachen, sodass der Abbau stattfinden kann, ohne dass beheimatete Tier- und Pflanzenarten bedroht oder die Sedimentschichten als Lebensraum zerstört werden. Gerade wird getestet, wie stark diese Methode den Lebensraum beeinflusst. \\
Weiterhin können Unterwassersonden und Roboter bei der Untersuchung und Reparatur von Offshore-Windanlagen eine große Erleichterung bieten. Besonders vor dem Hintergrund des Ausbaus der erneuerbaren Energien und der folglich steigenden Anzahl an Windparks vor den Küsten wird die Instandhaltung dieser Anlagen immer wichtiger. Dabei sind Effizienz und Vereinfachung wichtige Faktoren. So können die Anlagen beispielsweise durch Kameras und Positionssensorik 3D-visualisiert werden und auf Schäden untersucht werden. Dadurch können aufwendige Tauchgänge durch Menschen ersetzt werden. Selbst Reparaturen können durch Roboter ferngesteuert durchgeführt werden. \\
Dies sind nur zwei Anwendungsbeispiele, an denen man sieht, dass Unterwassertechnik ein immer wichtigeres Thema wird und daher die Forschung- und Entwicklungsarbeit in diesem Bereich bedeutsamen Fortschritt bringt.
}

\newcommand\Ziele{
    Wir planen, nicht nur einfach eine Sonde zu bauen, sondern eine Plattform für Schüler und Schülerinnen unserer Schule, auf der sie in Zukunft auch eigene Experimente durchführen können. Deshalb soll die Bedienung der Sonde auch unkompliziert sein. Eine robuste Konstruktion ist unerlässlich, damit die Sonde öfter verwendet werden kann und während des Tauchgangs stabil bleibt. Hierbei sollen bewegliche Teile zur Vereinfachung und Stabilität so weit wie möglich minimiert werden. \\
    Die Sonde soll die Fähigkeit besitzen, verschiedene Experimente je nach den aktuellen Anforderungen zu transportieren. Durch einen modularen Aufbau wird ermöglicht, dass zusätzliche Module, die eigene Experimente oder Sensorik beinhalten, gebaut und sehr einfach eingebaut werden können. \\
    Es ist wichtig, dass die Sonde die Fähigkeit hat, Elektronik in einem wasserdichten Raum zu transportieren. Die Sonde soll auch groß genug sein, damit die Größe unserer (Elektronik-)Module und auch die der im Nachhinein konstruierten Experimentiermodule wenig eingeschränkt werden. \\
    Die Sonde muss in der Lage sein, in einer Tauchtiefe von ca. 70 Metern unter dem Meeresspiegel in sowohl Süß- als auch Salzwasser zu arbeiten. Sie benötigt nur eine Art Tiefenkontrolle, da keine Bewegung in die anderen Richtungen stattfindet. Während des Tauchvorgangs sollen durch die Elektronik an Bord Daten (Druck, Temperatur, Licht, Neigung, GPS) und Video-/Soundaufnahmen aufgezeichnet, sowie Wasserproben gesammelt werden, die dann zurück an der Oberfläche entnommen und später analysiert werden können, um weitere Informationen über das Gewässer zu erlangen. \\
    Natürlich sind Kosten aber auch Nachhaltigkeit ein entscheidender Faktor bei der Konstruktion und der Aufwand für einen einzelnen Tauchgang soll möglichst geringgehalten werden.
}
\newcommand\Thema{
    Ab der zehnten Klasse gab es an unserer Schule das Angebot der freiwilligen AG „JIA“ (Junior Ingenieurs Akademie). Im Rahmen dieser AG wurden verschiedene technische Projekte verwirklicht (z.B. Formel 1 in der Schule). Die Gruppe aus der höheren Klasse, die damals vor uns ein Projekt entwickelte, war an unserer Schule relativ erfolgreich mit ihrem Projekt. Sie hatten damals einen Wetterballon steigen lassen, welcher verschiedene Daten während des Fluges aufnahm (genannt: „KGSgoesStrato“). Nun wurden wir von unseren Lehrern angeregt auch ein derartiges Projekt zu gestalten. Und so mussten wir uns ein passendes Thema für unser Projekt überlegen. Die Überlegung ging dann in die Richtung: „Das vorherige Projekt ging nach oben, wie wäre es, wenn wir jetzt nach unten gehen, vielleicht unter Wasser?“. Die Idee fand schnell Anklang und es wurde auch angemerkt, dass für den Biologieunterricht (und auch Geografieunterricht) oft Gewässerdaten und auch Wasserproben gesucht werden, um diese auszuwerten. So entstand die Idee, eine Sonde zu entwickeln, welche diese Aufgaben und mehr erfüllen kann.
}
\newcommand\KonteptEins{
    Das erste Konzept hatte einen Tauchmechanismus ähnlich dem, der in echten U-Booten verwendet wird, und würde modulare Experimentmodule tragen, die übereinandergestapelt werden.
    Der Tauchgang würde von einem speziellen Modul durchgeführt werden, welches die gesamte für den Tauchgang erforderliche Hardware enthält. Zur Tiefenkontrolle werden ein Wassertank und ein Luft Tank verwendet, die zusammen mit Hilfe computergesteuerter Ventile den Auf- und Abtrieb der Sonde kontrolliert. 
    Verschiedene Experimentiermodule werden übereinandergestapelt und miteinander verbunden. Diese Module können angepasst werden und ermöglichen es, je nach Bedarf wissenschaftliche Experimente zur Sonde hinzuzufügen oder zu entfernen. 
}